\section{Introduction}
Convolutional neural networks have become widely known as the state-of-the-art standard in image recognition - a field with numerous practical applications. One of these applications is identified by the "Makerere Passion Fruit Disease Detection Challenge" \cite{zindi}. The challenge - presented by The Marconi Society Machine Learning Laboratory at Makerere University - is addressing the classification of passion fruit diseases by developing a low-cost hand-held diagnostic device which makes use of state-of-the-art machine learning techniques. This report details the architecture, experimentation and results of a convolutional neural network inline with this initiative.

The report is structured such that we first provide the appropriate background on each of the aforementioned concepts, and on the data set. Then, we elaborate on the experimental setup - we detail any data pre-processing that was done, we describe the architecture of the networks, we explain why specific hyper-parameters were chosen for each of the methods, and we note the process utilized to perform experimentation. Penultimately, we present and discuss the results and, finally, we conclude with a summary of the findings and any additional remarks.